\subsection{查询语法}
\begin{frame}[allowframebreaks, fragile]{查询语句格式}
\begin{block}{查询语法}
\begin{lstlisting}[language=SQL]
SELECT [ALL|DISTINCT] <目标列表达式>
                                       [,<目标列表达式>] …
FROM <表名或视图名>[, <表名或视图名> ] …
[ WHERE <条件表达式> ]
[ GROUP BY <列名1> [ HAVING <条件表达式> ] ]
[ ORDER BY <列名2> [ ASC|DESC ] ];

\end{lstlisting}
\end{block}
\begin{itemize}
    \item SELECT子句:指定要显示的属性列
    \item FROM子句:指定查询对象(基本表或视图)
    \item WHERE子句:指定查询条件
    \item GROUP BY子句:对查询结果按指定列的值分组,该属性列值相等的元组为一个组。通常会在每组中作用集函数
    \item HAVING短语:筛选出只有满足指定条件的组
    \item ORDER BY子句:对查询结果表按指定列值的升序或降序排序 
\end{itemize}
\end{frame}


\subsection{单表查询}
\begin{frame}[fragile]{单表查询}
\begin{itemize}
    \item 查询仅涉及一个表,是一种最简单的查询操作
    \begin{itemize}
        \item 选择表中的若干列
        \item 选择表中的若干元组
        \item 对查询结果排序
        \item 使用集函数
        \item 对查询结果分组  
    \end{itemize}
\end{itemize}
\end{frame}

\begin{frame}[fragile]{选择表中的若干列}
\begin{itemize}
    \item 查询全体学生的学号与姓名
    \item 查询全体学生的姓名、学号、所在系
    \item 查询全体学生的详细记录
\end{itemize}

\begin{block}{单表查询例子}
\begin{lstlisting}[language=SQL]
SELECT Sno, Sname FROM Student;
SELECT Sname,Sno,Sdept FROM Student;
SELECT * FROM Student; 

Sno       Sname  Ssex  Sage  Sdept
--------  -----  ----  ----  -----
20231521  李勇     男     20    CS
20231522  刘晨     女     19    CS
20231523  王敏     女     18    MA
20231525  张立     男     19    AI

\end{lstlisting}
\end{block}
\end{frame}




\begin{frame}[fragile]{选择表中的若干列}
\begin{itemize}
    \item 查全体学生的姓名及其出生年份
    \item 查询全体学生的姓名、出生年份和所在系。在出生年份前面增加一个说明,在系名称后面增加一个“系”作为表示 (Hint: AS 别名)
\end{itemize}

\begin{block}{单表查询例子}
\begin{lstlisting}[language=SQL]
SELECT Sname,2023-Sage FROM Student;
SELECT Sname, 2023-Sage AS  '出生年份',  Sdept || '系' AS '院系' FROM Student;

Sname  出生年份  院系
-----  ----  ---
李勇     2003  CS系
刘晨     2004  CS系
王敏     2005  MA系
张立     2004  AI系

\end{lstlisting}
\end{block}
\end{frame}


\begin{frame}[fragile]{选择表中的若干元组}
\begin{itemize}
    \item 消除取值重复的行
    \begin{itemize}
        \item 查询选修了课程的学生学号
    \end{itemize}
    % \item 查询满足条件的元组 
    % \begin{itemize}
    %     \item 查询选修了课程的学生学号
    % \end{itemize}
\end{itemize}

\begin{block}{查询例子}
\begin{lstlisting}[language=SQL]
SELECT Sno FROM SC;
Sno
--------
20231521
20231521
20231521
20231522
20231522

SELECT DISTINCT Sno FROM SC;
Sno
--------
20231521
20231522

\end{lstlisting}
\end{block}
\end{frame}


\begin{frame}[fragile]{选择表中的若干元组}
\begin{itemize}
    \item 查询满足条件的元组 (WHERE)
    \begin{itemize}
        \item 运算符:=,>,<,>=,<=,!= 或 <>,!>,!<
        \item 关键词:NOT / BETWEEN …  AND / IN
    \begin{itemize}
        \item 查询所有年龄在20岁以下的学生姓名及其年龄
        \item 查询年龄在20~23岁(包括20岁和23岁)
        \item 查询年龄不在20~23岁之间的学生姓名、系别和年龄
        \item 查询信息系(IS)和计算机科学系(CS)学生的姓名和性别。
    \end{itemize}
    \end{itemize}
    
\end{itemize}

\begin{block}{查询例子}
\begin{lstlisting}[language=SQL]
SELECT Sname, Sage FROM Student WHERE Sage < 20;
SELECT Sname, Sdept, Sage FROM Student WHERE Sage BETWEEN 20 AND 23; 
SELECT Sname, Sdept, Sage FROM Student  WHERE Sage NOT BETWEEN 20 AND 23;
SELECT Sname, Ssex FROM  Student WHERE Sdept IN ( 'IS', 'CS'  );
\end{lstlisting}
\end{block}
\end{frame}



\begin{frame}[fragile]{选择表中的若干元组}
\begin{itemize}
    \item 字符串匹配:  [NOT] LIKE  ‘<匹配串>’ 
    \begin{itemize}
    \item \% (百分号)  代表任意长度(长度可以为0)的字符串
    \item \_ (下横线)  代表任意单个字符
    \begin{itemize}
        \item 查询所有姓刘学生的姓名、学号和性别
        \item 查询姓“刘"且全名为三个汉字的学生的姓名
        \item 查询所有不姓刘的学生姓名
    \end{itemize}
    \end{itemize}
    \item 使用谓词 IS NULL 或 IS NOT NULL
\begin{itemize}
        \item 有些课没有先修课程。查询没有先修课程的课程名称
    \end{itemize}
\end{itemize}

\begin{block}{查询例子}
\begin{lstlisting}[language=SQL]
SELECT Sname, Sno, Ssex FROM Student WHERE Sname LIKE '刘%';
SELECT Sname, Sno, Ssex FROM Student WHERE Sname LIKE '刘__';
SELECT Sname, Sno, Ssex FROM Student WHERE Sname NOT LIKE '刘%';
SELECT Cname From Course WHERE Cpno IS NULL;

\end{lstlisting}
\end{block}
\end{frame}


\begin{frame}[fragile]{选择表中的若干元组}
\begin{itemize}
    \item 对查询结果进行排序  (ORDER BY) 
    \begin{itemize}
        \item 可以按一个或多个属性列排序
        \item 升序:ASC;降序:DESC;缺省值为升序
        \begin{itemize}
        \item 查询选修了3号课程的学生的学号及其成绩,查询结果按分数降序排列
        \item 查询全体学生情况,查询结果按所在系的系号升序排列,同一系中的学生按年龄降序排列
        \end{itemize}
    \end{itemize}
    \item 使用集函数
    \begin{itemize}
        \item COUNT/ SUM / AVG / MAX/ MIN
        \begin{itemize}
            \item 查询选修了课程的学生人数
            \item 计算2号课程的学生平均成绩
        \end{itemize}
    \end{itemize}
    
\end{itemize}

\begin{block}{查询例子}
\begin{lstlisting}[language=SQL]
SELECT Sno,Grade FROM  SC WHERE  Cno= '3' ORDER BY Grade DESC;
SELECT * FROM  Student ORDER BY Sdept, Sage DESC; 
SELECT COUNT(DISTINCT Sno) FROM SC;
SELECT AVG(Grade) FROM SC WHERE Cno= '2';


\end{lstlisting}
\end{block}
\end{frame}




\begin{frame}[fragile]{选择表中的若干元组}
\begin{itemize}
    \item 使用GROUP BY子句分组  
    \begin{itemize}
        \item 未对查询结果分组,聚集函数将作用于整个查询结果
        \item 对查询结果分组后,聚集函数将分别作用于每个组 
        \begin{itemize}
        \item 求各个课程号及相应的选课人数
        \end{itemize}
    \end{itemize}
    \item 使用HAVING短语筛选最终输出结果
    \begin{itemize}
        \item 只有满足HAVING短语指定条件的组才输出
        \begin{itemize}
            \item 查询选修了2门及以上课程的学生学号
            \item 查询有2门及以上课程是85分以上的
          学生的学号及(85分以上的)课程数 
        \end{itemize}
    \end{itemize}
    
\end{itemize}

\begin{block}{查询例子}
\begin{lstlisting}[language=SQL]
SELECT Cno,COUNT(Sno) FROM SC GROUP BY Cno;
SELECT Sno FROM  SC GROUP BY Sno HAVING  COUNT(*) >2; 
SELECT Sno, COUNT(*) FROM   SC
    WHERE Grade>=85 GROUP BY Sno HAVING COUNT(*)>=2;


\end{lstlisting}
\end{block}
\end{frame}



\subsection{连接查询}
\begin{frame}[fragile]{连接查询}
\begin{itemize}
    \item 若一个查询同时涉及两个以上的表,称为连接查询
    \begin{itemize}
        \item 等值连接 / 非等值连接 / 
        \begin{itemize}
            \item 语法
\begin{block}{}
            \begin{lstlisting}[language=SQL]
[<表名1>.]<列名1>  <比较运算符>  [<表名2>.]<列名2>
            \end{lstlisting}
\end{block}
            
            \item 比较运算符:=、>、<、>=、<=、!=
            \item 查询每个学生及其选修课程的情况
        \end{itemize}
        \item 自然连接: 等值连接的一种特殊情况,把目标列中重复的属性列去掉
    \end{itemize}
\end{itemize}
\begin{block}{查询例子}
\begin{lstlisting}[language=SQL]
SELECT  Student.*, SC.* FROM Student, SC WHERE  Student.Sno = SC.Sno;
\end{lstlisting}
\end{block}
\end{frame}

\begin{frame}[fragile]{连接查询}
\begin{itemize}
    \item  若一个查询同时涉及两个以上的表,称为连接查询
 
    \begin{itemize}
        \item 一个表与其自己进行连接,称为表的自身连接
        \begin{itemize}
            \item 需要给表起别名以示区别
            \item 由于所有属性名都是同名属性,因此必须使用别名前缀
            \item 查询每一门课的间接先修课(即先修课的先修课)
        \end{itemize}
        \item 外连接操作以指定表为连接主体,将主体表中不满足连接条件的元组一并输出
        \begin{itemize}
            \item  查询每个学生及其选修课程的情况
包括没有选修课程的学生----用外连接操作

        \end{itemize}

    \end{itemize}
\end{itemize}
\begin{block}{查询例子}
\begin{lstlisting}[language=SQL]
SELECT  FIRST.Cno, SECOND.Cpno FROM  Course  FIRST, Course  SECOND WHERE FIRST.Cpno = SECOND.Cno; 

SELECT  Student.Sno, Sname, Ssex, Sage, Sdept, Cno, Grade FROM Student 
LEFT JOIN SC  ON (Student.Sno=SC.Sno);
\end{lstlisting}
\end{block}
\end{frame}


\begin{frame}[fragile]{连接查询}
\begin{itemize}
    \item  若一个查询同时涉及两个以上的表,称为连接查询
 
    \begin{itemize}
        \item WHERE子句中含多个连接条件时,称为复合条件连接
        \begin{itemize}
            \item 查询选修2号课程且成绩在86分以上的所有学生的学号、姓名
            \item 查询每个学生的姓名、选修课程名及成绩。
        \end{itemize}

    \end{itemize}
\end{itemize}
\begin{block}{查询例子}
\begin{lstlisting}[language=SQL]
SELECT Student.Sno, student.Sname FROM Student, SC
WHERE Student.Sno = SC.Sno AND  SC.Cno= '2' AND  SC.Grade > 86;

SELECT Sname,Cname,Grade FROM Student, SC, Course
WHERE Student.Sno = SC.Sno AND SC.Cno = Course.Cno;

\end{lstlisting}
\end{block}
\end{frame}

\subsection{嵌套查询}
\begin{frame}[fragile]{嵌套查询}
\begin{itemize}
    \item  一个SELECT-FROM-WHERE语句称为一个查询块
    \item 将一个查询块嵌套在另一个查询块的WHERE子句或HAVING短语的条件中的查询称为嵌套查询
    
    \begin{itemize}
        \item 子查询的限制 (不能使用ORDER BY子句)
        \item 层层嵌套方式反映了 SQL语言的结构化
        \item 有些嵌套查询可以用连接运算替代
        \begin{itemize}
            \item 查询选修2号课程的学生信息。
        \end{itemize}

    \end{itemize}
\end{itemize}
\begin{block}{查询例子}
\begin{lstlisting}[language=SQL]
SELECT Sname FROM Student WHERE Sno IN // 父查询
(SELECT Sno FROM SC WHERE Cno= '2'); //子查询
\end{lstlisting}
\end{block}
\end{frame}

\begin{frame}[fragile]{嵌套查询}
\begin{itemize}
    \item  带有IN谓词的子查询
    \begin{itemize}
        \item 例子
        \begin{itemize}
            \item 查询与“刘晨”在同一个系学习的学生
            \item 查询选修了课程名为“信息系统”的学生学号和姓名
            \item 找出每个学生超过他自己选修课程平均成绩的课程号
        \end{itemize}

    \end{itemize}
\end{itemize}
\begin{block}{查询例子}
\begin{lstlisting}[language=SQL]
SELECT Sno, Sname, Sdept FROM Student WHERE Sdept  IN
(SELECT Sdept FROM Student WHERE Sname= '刘晨');

SELECT Sno, Sname FROM Student WHERE Sno IN
(SELECT Sno FROM SC WHERE  Cno IN 
    (SELECT Cno FROM Course WHERE Cname= '信息系统'));

\end{lstlisting}
\end{block}
\end{frame}

\begin{frame}[fragile]{嵌套查询}
\begin{itemize}
    \item 带有比较运算符的子查询
    \item 带有ANY或ALL谓词的子查询
    \item 带有EXISTS谓词的子查询
    \begin{itemize}
        \item 例子
        \begin{itemize}
            \item 找出每个学生超过他自己选修课程平均成绩的课程号
            \item 查询其他系中比信息系某一个(其中某一个)学生年龄小的学生姓名和年龄
            \item 查询所有选修了2号课程的学生姓名
        \end{itemize}

    \end{itemize}
\end{itemize}
\begin{block}{查询例子}
\begin{lstlisting}[language=SQL]
SELECT Sno, Cno From SC x WHERE Grade >= (select avg(Grade) from SC y where y.Sno=x.Sno);
SELECT Sname, Sage FROM Student WHERE Sage<ANY (SELECT  Sage FROM  Student  WHERE Sdept='IS') AND Sdept != 'IS' ;  
SELECT Sname FROM Student WHERE EXISTS 
(SELECT * FROM SC WHERE Sno=Student.Sno AND Cno= '2');

\end{lstlisting}
\end{block}
\end{frame}


\subsection{集合查询}
\begin{frame}[fragile]{集合查询}
\begin{itemize}
    \item 集合查询的操作
    \begin{itemize}
        \item 并操作(UNION)
        \item 交操作(INTERSECT)
        \item 差操作(EXCEPT)
        \begin{itemize}
            \item 查询计算机科学系的学生及年龄不大于19岁的学生
            \item 查询选修课程1的学生集合与选修课程2的学生集合的交集
            \item 查询学生姓名与教师姓名的差集
        \end{itemize}

    \end{itemize}
\end{itemize}
\begin{block}{查询例子}
\begin{lstlisting}[language=SQL]
SELECT * FROM Student  WHERE Sdept= 'CS' UNION
SELECT * FROM Student WHERE Sage<=19;

SELECT Sno FROM SC WHERE Cno='1' AND 
Sno IN (SELECT Sno FROM SC WHERE Cno='2');

SELECT DISTINCT Sname FROM Student
WHERE Sname NOT IN (SELECT Tname FROM Teacher);

\end{lstlisting}
\end{block}
\end{frame}

