\subsection{定义基本表}
\begin{frame}[fragile]{定义基本表}
\begin{itemize}
    \item 基本表的定义 (CREATE)
    \item 数据类型
    \begin{itemize}
        \item 在SQL中域的概念用数据类型实现
        \item 属性数据类型的确定:如年龄、学号
        \item 数据类型:
            \begin{itemize}
                \item 字符串:CHAR, VARCHAR, TEXT
                \item 数值: INT, FLOAT, DOUBLE
                \item 其他:DATE, BOOLEAN, BLOB
            \end{itemize}
    \end{itemize}
\end{itemize}
\begin{block}{创建表语法}
\begin{lstlisting}[language=SQL]
CREATE TABLE <表名> ( 
    <列名> <数据类型> [列完整性约束] 
    [, <列名> <数据类型> [列完整性约束]]
    ...\\
    [, <表完整性约束]);
\end{lstlisting}
\end{block}
\end{frame}

\begin{frame}[fragile]{定义基本表}
\begin{itemize}
    \item 基本表的定义 (CREATE)
    \item 完整性约束
    \begin{itemize}
        \item 主码约束: PRIMARY KEY
        \item 唯一性约束: UNIQUE
        \item 非空值约束: NOT NULL
        \item 参照完整性约束: REFERENCES
    \end{itemize}
\end{itemize}
\begin{block}{创建表语法}
\begin{lstlisting}[language=SQL]
CREATE TABLE <表名> ( 
    <列名> <数据类型> [列完整性约束] 
    [, <列名> <数据类型> [列完整性约束]]
    ...\\
    [, <表完整性约束]);
\end{lstlisting}
\end{block}
\end{frame}



\begin{frame}[fragile]{建立学生表}
\begin{itemize}
    \item 建立一个“学生”表Student,它由学号Sno、姓名Sname、性别Ssex、年龄Sage、所在系Sdept五个属性组成。其中学号是主键,并且姓名取值也唯一
\end{itemize}

\begin{block}{建立学生表}
\begin{lstlisting}[language=SQL]
CREATE TABLE Student(
   Sno CHAR(10) PRIMARY KEY,
   Sname CHAR(5) UNIQUE,
   Ssex CHAR(2),
   Sage INT,
   Sdept CHAR(20)
);
\end{lstlisting}
\end{block}  
\end{frame}


\begin{frame}[fragile]{建立课程表}
\begin{itemize}
    \item 建立一个“课程”表Course,它由课程号Cno、课程名Cname、先修课程号Cpno、学分Ccredit组成。其中课程号是主键
\end{itemize}

\begin{block}{建立课程表}
\begin{lstlisting}[language=SQL]
CREATE TABLE Course(
    Cno CHAR(4) PRIMARY KEY,
    Cname CHAR(40),
    Cpno CHAR(4),
    Ccredit INT,
    FOREIGN KEY (Cpno) REFERENCES Course(Cno)
);
\end{lstlisting}
\end{block}
    
\end{frame}


\begin{frame}[fragile]{建立选课表}
\begin{itemize}
    \item 建立一个“学生选课”表SC,它由学号Sno、课程号Cno,修课成绩Grade组成,其中(Sno, Cno)为主码。
\end{itemize}

\begin{block}{建立选课表}
\begin{lstlisting}[language=SQL]
CREATE TABLE SC(
   Sno CHAR(10),
   Cno CHAR(4),
   Grade INT,
   PRIMARY KEY (Sno, Cno),
   FOREIGN KEY (Sno) REFERENCES Student(Sno),
   FOREIGN KEY (Cno) REFERENCES Course(Cno),
);
\end{lstlisting}
\end{block}
    
\end{frame}


\subsection{修改基本表}

\begin{frame}[fragile]{修改基本表}
\begin{itemize}
    \item 修改基本表 ALTER
    \item 表名:要修改的基本表
    \begin{itemize}
        \item ADD子句:增加新列和新的完整性约束条件
        \item DROP子句:删除指定的完整性约束条件
        \item ALTER COLUMN子句:用于修改列名和数据类型
        \item DROP COLUMN子句:用于删除列
    \end{itemize}
\end{itemize}
\begin{block}{修改表语法}
\begin{lstlisting}[language=SQL]
ALTER TABLE <表名>
    [ ADD <新列名> <数据类型> [ 完整性约束 ] ]
    [ DROP <完整性约束名> ]
    [ALTER COLUMN <列名> <数据类型> ]
    [DROP COLUMN <列名> <数据类型> ];
\end{lstlisting}
\end{block}
\end{frame}


\begin{frame}[fragile]{修改基本表例子}
\begin{itemize}
    \item 向Student表增加“入学时间”列,其数据类型为日期型
    \item 将入学日期的数据类型改为字符型
    \item 将入学日期列删除掉
    \item 删除选课表
\end{itemize}
\begin{block}{修改表语法}
\begin{lstlisting}[language=SQL]
ALTER TABLE Student ADD Scome DATETIME;
ALTER TABLE Student ALTER COLUMN Scome char(8);
ALTER TABLE Student DROP COLUMN Scome;
DROP TABLE SC;
\end{lstlisting}
\end{block}
\end{frame}

\begin{frame}[fragile]{建立和删除索引}
\begin{itemize}
    \item 建立索引是加快查询速度的有效手段
    \item 建立索引
    \begin{itemize}
        \item DBA或表的属主(即建立表的人)根据需要建立
        \item 有些DBMS自动建立以下列上的索引 ( PRIMARY  KEY, UNIQUE)
    \end{itemize}
    \item 维护索引: DBMS自动完成
    \item 使用索引: DBMS自动选择是否使用索引以及使用哪些索引
\end{itemize}
\begin{block}{修改表语法}
\begin{lstlisting}[language=SQL]
CREATE [UNIQUE] [CLUSTER] INDEX <索引名>     ON <表名>(<列名>[<次序>][,<列名>[<次序>] ]…);
DROP INDEX <索引名>;
\end{lstlisting}
\end{block}
\end{frame}

\begin{frame}[fragile]{建立和删除索引例子}
\begin{itemize}
    \item 为学生-课程数据库中的Student,Course,SC三个表建立索引。其中Student表按学号升序建唯一索引,Course表按课程号升序建唯一索引,SC表按学号升序和课程号降序建唯一索引
    \item 删除Student表的SCno索引
\end{itemize}
\begin{block}{修改表语法}
\begin{lstlisting}[language=SQL]
CREATE UNIQUE INDEX  Stusno ON Student(Sno);
CREATE UNIQUE INDEX Coucno ON Course(Cno);
CREATE UNIQUE INDEX SCno ON SC(Sno ASC, Cno DESC); 
DROP INDEX SCno;
\end{lstlisting}
\end{block}
\end{frame}



\subsection{插入数据}
\begin{frame}[fragile]{插入数据}
\begin{itemize}
    \item 使用 INSERT INTO 语句插入数据
\end{itemize}

\begin{block}{插入数据}
\begin{lstlisting}[language=SQL]
INSERT INTO Student VALUES ('20231521', '李勇', '男', 20, 'CS');
INSERT INTO Student VALUES ('20231522', '刘晨', '女', 19, 'CS');
INSERT INTO Student VALUES ('20231523', '王敏', '女', 18, 'MA');
INSERT INTO Student VALUES ('20231525', '张立', '男', 19, 'AI');
INSERT INTO Course VALUES ('1', '数据库', '5', 4);
INSERT INTO Course VALUES ('2', '数学', NULL , 2);
INSERT INTO Course VALUES ('3', '信息系统', 1 , 4);
INSERT INTO Course VALUES ("4", '操作系统', 6, 3);
INSERT INTO Course VALUES ("5", '数据结构', 7, 4);
INSERT INTO Course VALUES ("6", '数据处理', NULL, 2);
INSERT INTO Course VALUES ("7", 'Python语言', NULL, 4);
INSERT INTO SC VALUES ("20231521", '1', 92);
INSERT INTO SC VALUES ("20231521", '2', 85);
INSERT INTO SC VALUES ("20231521", '3', 88);
INSERT INTO SC VALUES ("20231522", '2', 90);
INSERT INTO SC VALUES ("20231522", '3', 80);
\end{lstlisting}
\end{block}
\end{frame}

